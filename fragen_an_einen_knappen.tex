%%This is a very basic article template.
%%There is just one section and two subsections.
\documentclass[twocolumn, pdftex, 12pt]{book}

\RequirePackage[english,ngerman]{babel}                    % deutsche Trennmuster
\RequirePackage[T1]{fontenc}                               % EC-Schriften, Trennstellen nach Umlauten
\RequirePackage[utf8]{inputenc}                          % direkte Umlauteingabe
% (� statt "a)

\begin{document}

\title{Fragen an einen Knappen \\ Eine Sammlung von Aufgaben zur Unterweisung
der höchst ritterlichen Tugenden.}

\author{Siggard Otto von Chevaccio-Guardiano zu Hohentann, \\ ordentliches
Mitglied des höchst tugendhaften Bundes zu Kron und Ring, \\ Ritter, Reisender
und Sucher der Tugend.}

\date{Spätsommer}

\maketitle

\chapter{Einleitung}

Im folgenden möchte ich eine Reihe von Situationen und Gegebenheiten vorstellen,
wie ich sie auf meinen Reisen erfahren habe oder von anderen Rittern und weisen
Männern vernommen habe. Einige habe ich mit eigenen Augen gesehen, andere wurden
berichtitet das selbige Ehrenmänner sie selbst gesehen haben. Einige waren als
Fragen zum Verständnis der Tugenden an mich gestellt, von denen jenen die mehr
Kenntnis dieser Dinge habe, oder von meine Vater als ich noch ein Kinde war.

Da ein Buch für einen Knappen, der des Lesen kundig sein sollte, nicht verborgen
sein wird und es dem Ritter, der dieses Buch besitze gefallen mag, es dem
Knappen zum Verständnis zu leihen, enthält es keine Antworten auf die Fragen.
Wer ein Ritter ist, wird eine rechte Antwort für jede Frage kennen.

\chapter{Ein Helm}

Es sei ein Turnier auf dem viele Ritter gar herlich streiten. Am Rande der
Schranken sind gar viele schöne Zelte aufgestellt, so das Herren und Damen
ebenso wie das Volk im Schatten den Kämpfen beiwohnen können. Unter diese
Dächern haben fleißig Knechte den eigenen Stuhl eines jeden Streiters gestellt,
so das dieser vom Kampfe ein wenig ausruhen mag und einen Krug Bier oder ein
Glas Wein trinken könne.

Ritter Arngrimm begibt sich zu seinem Stuhl und findet darauf einen gar
prächtigen Helm, verziehrt in Gold und bedeckt mit Gravur und Ätzung auf
prächtige Weise. Doch ist es ist nicht sein eigener. Auch
ist kein Besitzer des Helmes oder dessen Knappe oder Knecht zu sehen. Damit kein
niederträchtiger Schuft diesen Helm stehlen mag, heist Arngrimm seinen Knappen
den Helm in Arngrimms Zelt in sicher zu verwahren. 

Nach einer kurzen Weile treffen Ritter Bernward und sein Knappe ein.
Es ist Bernwards Helm und er ist erzürnt das scheinbar ein Schuft das gute
Stück entwendet hat. Arngrimm ist wendet sich erbost an Bernward, weil er sich
nicht auf seinem Stuhl ausruhen konnte, ohne vorher für die Sicherheit des
Helmes zu sorgen. Arngrimm sagt das der Helm in seinem Zelt ist.

Von seinem Knappen weiß Bernward dieser den Helm auf den Stuhl des Arngrim
gelegt hat.

\emph{Du, braver Knappe sollst nun für den Augenblick der edle Bernward sein. Was
tust und sagst du?}

\chapter{Kein Zweig}

Ritter Tankwart ist belehnt von Graf Sigismund mit einem reichen und prächtigen
Lehen, der Gemarkung Ernfels. Sie umfasst sieben Dörfer und einen Marktflecken.
Seit mehreren Jahren sind die Ernten gut und kein Edler trachtet nach dem
Rittergut oder dem Land seiner Gnaden, so das keine Fehden gefochten werden
mussten. Eines Tages lässt seine Gnaden Tankwart zu sich kommen.

``Tankwart, ich habe erfahren von einer schrecklichen Krankheit, die dein Dorf
Grünau befallen hat.'' spricht der Graf. ``Gehe hin und erschlage jeden Mann,
jede Frau und jedes Kind in diesem Dorfe. Verschone niemanden. Danach
entzünde jedes Haus. Der Segen Thalans ist über dir und er wird dich vor der
Krankheit schützen. Dies ist mein Wunsch und Wille''.

\emph{Du, braver Knappe sollst nun für den Augenblick der edle Tankwart sein.
Was tust und sagst du?}

\chapter{Eine Weggabelung}

Friedhelm ist reisender Ritter und wandert auf einer einsamen Straße und reitet
durch einen Holhlweg. Der Hohlweg öffnet sich auf einer Lichtung auf der die
Straße in mehrere Richtungen verzweigt. An jeder der anderen Straßen und Wege
die von der Lichtung weiter führen steht eine Person, die von mehreren Räubern
und Haderlumpen bedroht wird. 

An einem Weg steht ein Ritter, der seine Waffe bereits gezogen hat.

Am nächsten Weg steht ein Kaufmann vor seinem Wagen.

Am dritten Weg steht ein Bauer vor seinem Ochsenkarren.

Die kleinen Bäume und Büsche auf der Lichtung verhindern das Friedhelm in die
Mitte der Lichtung reiten kann um die Aufmerksamkeit aller Spitzbuben auf sich
zu lenken. 

\emph{Du, braver Knappe sollst nun für einen Augenblick der edle Friedhelm
sein. Du kannst nur einem der drei helfen. Was tust du?}

\end{document}
